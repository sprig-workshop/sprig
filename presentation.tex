\documentclass{beamer}
%
% Choose how your presentation looks.
%
% For more themes, color themes and font themes, see:
% http://deic.uab.es/~iblanes/beamer_gallery/index_by_theme.html
%
\mode<presentation>
{
  \usetheme{default}      % or try Darmstadt, Madrid, Warsaw, ...
  \usecolortheme{seahorse} % or try albatross, beaver, crane, ...
  \usefonttheme{structurebold}  % or try serif, structurebold, ...
  \setbeamertemplate{navigation symbols}{}
  \setbeamertemplate{caption}[numbered]
}

\usepackage[english]{babel}
\usepackage[utf8x]{inputenc}

\title[Your Short Title]{Your Presentation}
\author{You}
\institute{Where You're From}
\date{Date of Presentation}

\begin{document}

\begin{frame}
  \titlepage
\end{frame}

% Uncomment these lines for an automatically generated outline.
%\begin{frame}{Outline}
%  \tableofcontents
%\end{frame}

%\section{Introduction}

%\begin{frame}{Introduction}

%\begin{itemize}
%\end{itemize}
%  \item Your introduction goes here!
%  \item Use \texttt{itemize} to organize your main points.
%
%\vskip 1cm

%\begin{block}{Examples}
%Some examples of commonly used commands and features are included, to help you get started.
%\end{block}

%\end{frame}

\begin{frame}{todo}
\begin{itemize}
  \item what does FN say about privacy?
  \item
  \item
\end{itemize}
\end{frame}


\begin{frame}{Manage your own Reputation}
\begin{itemize}
  \item How we are judged by others affects our opportunities, friendships, and overall well-being.
  \item Privacy enables us to control our reputation to some degree. Choosing what we want to share, with who with the right context is essential as we want to be judged fairly.
  \item
\end{itemize}
\begin{block}{Question}
Have you ever experienced someone sharing one of your secret to others? \\
A crush you had?
\end{block}
\end{frame}


\begin{frame}{Respect for the Individual}
\begin{itemize}
  \item Privacy is about respecting the individual
  \item United Nations states in article 12 that "No one shall be subjected to arbitrary interference with his privacy, family, home or correspondence, nor to attacks upon his honor and reputation.", and that one is protected by the law against such attacks.
\end{itemize}
\begin{block}{Question}
How did it feel when someone shared one of your secrets?
\end{block}
\end{frame}


\begin{frame}{Limit Power}
\begin{itemize}
  \item The more someone knows about us, the more power they can have over us.
  \item Personal data can affect our reputations, and it can be used to influence our decisions and shape our behavior.
  \item Personal data can cause great harm and be used to control us.
\end{itemize}
\begin{block}{Question}
Have you experienced that someone have threatened to tell something to someone that you did not want to share? That you liked someone?
\end{block}
\end{frame}



\begin{frame}{Trust}
\begin{itemize}
  \item In all our relationships, personal, professional, and others we depend on trusting the other party.
  \item Trust allows us to be open and honest in relationships.
\end{itemize}
\begin{block}{Question}
Have you experienced having your trust broken?
\end{block}
\end{frame}


\begin{frame}{Social Boundaries}
\begin{itemize}
  \item We learn from each other what is acceptable behavior. This lets us know what boundaries we should stay inside.
  \item We need to be able to access to silence and solitude, places to relax and escape from the gaze of others.
  \item As everyone is affecting the boundaries together each of our relations to privacy mater. Books like "1981" and "The Circle" illustrates this.
\end{itemize}
\begin{block}{Question}
Have you ever told someone "To much information" or "None of your business"?
\end{block}
\end{frame}





\begin{frame}{Control over one's life}
\begin{itemize}
  \item To be free one needs to have control over one's life
  \item personal data affects decisions about us.Do we get a loan? a license? a job?
  \item Knowing what data is used and how, as well as being able to correct the data is necessary to be in control over one's life in the digital age.
\end{itemize}
\begin{block}{Question}
Have you experienced that some information about you have been incorrect without being able to correct it? Or that some information you have shared have been used in an unexpected way?
\end{block}
\end{frame}


\begin{frame}{Freedom of Thought and Speech}
\begin{itemize}
  \item Without privacy
  \item
  \item from \url{https://teachprivacy.com/10-reasons-privacy-matters/}
\end{itemize}
\begin{block}{Question}

\end{block}
\end{frame}


\begin{frame}{Social Boundaries}
\begin{itemize}
  \item
\end{itemize}
\begin{block}{Question}

\end{block}
\end{frame}




\end{document}
